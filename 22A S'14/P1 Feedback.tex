\documentclass[8pt]{article}

\usepackage{graphicx}
\usepackage{amsmath,amsfonts,amsthm,amssymb}
\usepackage[margin=.9in]{geometry}
%\topmargin=-0.45in      %
%\evensidemargin=0in     %
%\oddsidemargin=0in      %
%\textwidth=6.5in        %
%\textheight=9.0in       %
%\headsep=0.25in         %


\newcommand{\ans}[1]
  {\noindent\fbox{\begin{minipage}[c]{\columnwidth}#1\end{minipage}}}


\newcommand{\ip}[1]{\left\langle #1 \right\rangle}
\newcommand{\torus}{\mathbb{T}}
\newcommand{\RRR}{\mathbb{R}}
\newcommand{\NNN}{\mathbb{N}}
\newcommand{\ZZZ}{\mathbb{Z}}
\newcommand{\QQQ}{\mathbb{Q}}
\newcommand{\CCC}{\mathbb{C}}
\newcommand{\HHH}{\mathcal{H}}
\newcommand{\FFF}{\mathcal{F}}

\newcommand{\B}[1]{\mathbf{ #1 }}



\makeatletter
\newcommand{\V}[2][r]{%
  \gdef\@VORNE{1}
  \left[\hskip-\arraycolsep%
    \begin{array}{#1}\vekSp@lten{#2}\end{array}%
  \hskip-\arraycolsep\right]}

\def\vekSp@lten#1{\xvekSp@lten#1;vekL@stLine;}
\def\vekL@stLine{vekL@stLine}
\def\xvekSp@lten#1;{\def\temp{#1}%
  \ifx\temp\vekL@stLine
  \else
    \ifnum\@VORNE=1\gdef\@VORNE{0}
    \else\@arraycr\fi%
    #1%
    \expandafter\xvekSp@lten
  \fi}
\makeatother






\begin{document}

\begin{center}
\textbf{Program 1 Feedback}
\end{center}

You all did pretty well on the first programming assignment but there is some room for improvement.  The following is a list of mostly organizational things I'd like you to pay attention to when turning in your future assignments.  This will not just help me as a grader, but you in organizing your work.\\

Recall that style counts for about 20 \% of your grades on these assignments so it is IMPORTANT that you follow them.  I graded easily on style this time, but in the future will take off points if you don't make an effort to follow the criteria below.\\

 \begin{enumerate}


\item \textbf{Style}\\


\begin{enumerate}
\item  Don't make your paper long for no good reason.  One of the best papers for Program One was around 8 pages, including code.  A longer paper does not mean a better grade, so try to be concise in your writing.

\item Arrange your work in the following fashion\\


problem 1\\
 \#1 analysis\\
\#1 programming code\\
 \#1 results\\ \\
 
 problem 2\\
 \#2 analysis\\
 \#2 programming code\\
 \#3 results\\
 
 and so on.  A reader should not have to search your entire report to find the answer to one problem.\\
 
 
\item When discussing results from your code make sure that the results are displayed close by.  You should not have a discussion of results on page 1 when the results are listed on page 10.  \\ \\
 
\end{enumerate}
 
 
 \item 
 \textbf{Analysis}\\
 
 \begin{enumerate}
 
 \item
 You do not need to derive a method or show proofs for things done in class.  When using a method in the assignment you should state it in a concise manner.\\
 
 For example you if using Euler's Method you could say something like:\\
 
 
 Euler's Method is a O(h) accurate method given by
 \[
 y_{i+1} = y_i + h*f(y_i, t_i)\\ \\
 \]
 
 
 The point of these assignments is for you to use these methods, not to derive them.\\
 
 \item  If you do not know how to typeset mathematics it is okay to leave a blank space and write the math in by hand.\\
 
 This is an example of what should NOT be done\\
 
"we solve the equation differential equation dy/dx = 4yx by separation of variables, first we multiply both sides by dx to get dy = 4y x dx then we divide both sides by y to get 1/y dy = 4 x dx and then we integrate to get $\ln(y) = 2x^2$, then we exponential both sides to get $y = e^(2x^2)$"\\

Nobody wants to read this.\\


\end{enumerate}

\item \textbf{Programming}\\


\begin{enumerate}


\item It is perfectly fine if you want to code in C++ or Python or whatever language you want, but I can only help you with coding issues if you use Matlab, R, or Octave.\\

I would urge you to try to use either Malab (free on math department computers), R (free to download), or Octave (free to download), as these are programming languages designed to be used for exactly the kinds of things we are doing in this class.  \\

Those of you who coded in a language besides Matlab had code that was both much longer, harder to read, and more difficult to code then if you had used Matlab.  \\

As the class progresses our codes will only get more complicated and you may find yourself having more trouble with writing the code then doing anything to do with math.\\

If you are worried about having to learn a new language, you should not worry! If you already know a programming language, it is easy to switch and you can come to Office Hours if you need help.\\ \\


\item  Your code should be well documented.  When I look at a section of code it should be obvious what the code is doing.  A reader doesn't need to understand what every variable represents, just the major ones.

A reader should look at a section of code and be able to say "this is such and such Method".\\


\item When displaying your code results make the data as easy to read as possible.  A reader should not have to read a paragraph below a picture to understand a picture.  All graphs should have labelled axis, a title that says what it displays, and if necessary a caption.  \\

A reader shouldn't look at a graph and wonder "What method is this? What is the resolution?"\\

\item Try to condense data as much as possible, if you run your method at 5 different resolutions, you should try not to have 5 graphs.  Instead, you could have them listed on the same graph with a legend.  \\

If you can't combine 5 different graphs into one, you shouldn't print them on 5 different pages, 5 graphs on one page is MUCH easier to read then 5 graphs on 5 pages.

\end{enumerate}

\item \textbf{Results}\\

\begin{enumerate}

\item Results are not simply the output of your computer code.  Results are taking your code's output and saying something worthwhile about it.  \\

Many of you listed the results of your code and said "as h gets smaller the solution gets more accurate"--that is NOT enough.  

You need to answer the question of how accurate it is in a mathematical fashion (just looking at it isn't good enough) and whatever other questions are asked by the assignments.



\end{enumerate}


and last but not least... START EARLY! You should be starting these assignments at least a week ahead of time, it is impossible to get it done the night before.



\end{enumerate}












\end{document}