\documentclass[8pt]{article}

\usepackage{graphicx}
\usepackage{amsmath,amsfonts,amsthm,amssymb}
\usepackage[margin=.9in]{geometry}
%\topmargin=-0.45in      %
%\evensidemargin=0in     %
%\oddsidemargin=0in      %
%\textwidth=6.5in        %
%\textheight=9.0in       %
%\headsep=0.25in         %


\newcommand{\ans}[1]
  {\noindent\fbox{\begin{minipage}[c]{\columnwidth}#1\end{minipage}}}


\newcommand{\ip}[1]{\left\langle #1 \right\rangle}
\newcommand{\torus}{\mathbb{T}}
\newcommand{\RRR}{\mathbb{R}}
\newcommand{\NNN}{\mathbb{N}}
\newcommand{\ZZZ}{\mathbb{Z}}
\newcommand{\QQQ}{\mathbb{Q}}
\newcommand{\CCC}{\mathbb{C}}
\newcommand{\HHH}{\mathcal{H}}
\newcommand{\FFF}{\mathcal{F}}

\newcommand{\B}[1]{\mathbf{ #1 }}



\makeatletter
\newcommand{\V}[2][r]{%
  \gdef\@VORNE{1}
  \left[\hskip-\arraycolsep%
    \begin{array}{#1}\vekSp@lten{#2}\end{array}%
  \hskip-\arraycolsep\right]}

\def\vekSp@lten#1{\xvekSp@lten#1;vekL@stLine;}
\def\vekL@stLine{vekL@stLine}
\def\xvekSp@lten#1;{\def\temp{#1}%
  \ifx\temp\vekL@stLine
  \else
    \ifnum\@VORNE=1\gdef\@VORNE{0}
    \else\@arraycr\fi%
    #1%
    \expandafter\xvekSp@lten
  \fi}
\makeatother






\begin{document}

\begin{center}
\textbf{HW2 Solutions, M22A, Spring Quarter 2014, Prof. Sornborger} \\
2.4: 2,5,6,11,14,17,21,26,31\\
2.5: 1,2,7,13,22,23,27,29\\
2.6: 1,3,4,5,7,10,12,15\\
2.7: 1,2,6,12,15,16
\end{center}


\textbf{2.4}: 2,5,6,11,14,17,21,26,31


\begin{enumerate}

\item[2] A is a 3x5 matrix, B and 5x3, C a 5x1, and D is 3x1.

What rows or columns of matrices do you multiply to find
\begin{enumerate}
\item the third column of AB

\item the first row of AB

\item the entry in row 3, column 4 of AB

\item the entry in row 1, column 1 of CDE

\end{enumerate}







\item[5]
Compute $A^2$ and $A^3$, make a predicition for $A^5$ and $A^n$.
\[
A = \V{ 1,b; 0,1}
\]
and
\[
A = \V{2,2;0,0}
\]







\item[6] Show that $(A+B)^2$ is different from $A^2 + 2AB + B^2$ when
\[
A = \V{1,2;0,0}, \ \ \ B = \V{1,0;3,0}
\]






\item[11] Choose the only B so that every matrix A
\begin{enumerate}
\item BA = 4A

\item BA = 4B

\item BA has rows 1 and 3 of A reverse and row 2 unchanged

\item All rows of BA are the same row 1 of A.
\end{enumerate}





\item[14] True or false:
\begin{enumerate}
\item if $A^2$ is defined then $A$ is necessarily quire

\item if AB and BA are defined the A and B are square

\item  If AB and BA are defined then AB and BA are square


\item  If AB = B then A =I
\end{enumerate}










\item[17]

Write down a 3 by 3 matrix A whose entries are

\begin{enumerate}
\item $a_{ij}$ = minimum of i and j

\item $a_{ij} = (-1)^{i + j}$

\item $a_{ij} = 1/j$

\end{enumerate}






\item[21]  Find all the powers $A^2, A^3,  . . . $ and $AB, (AB)^2 . . .$ for 
\[
A = \V{.5,.5;.5,.5}, \ \ \ B = \V{1,0;0,-1}
\]









\item[26]

Multiply AB using columns times rows
\[
AB = \V{1,0;2,4;2,1} \V{3,3,0;1,2,1} = \V{1;2;2} \V{3,3,0} +BLANK = BLANK
\]








\item[31]

With $i^2 = -1$ the product of of $A + iB$ and $x + i y$ is $Ax + iBx + iAy - By$.  Use blocks to separate the real part without i from the imaginary party that multiplies i

\[
\V{A, -B; ? , ?} \V{x;y} = \V{Ax - By;?}
\]

\end{enumerate}

\newpage










\textbf{2.5}: 1,2,7,13,22,23,27,29




\begin{enumerate}

\item[1]  Find the inverses of A,B,C
\[
A = \V{0,3;4,0}, \ \ \ B = \V{2,0;4,2}, \ \ \ C = \V{3,4;5,7}
\]

\item[2]

For these permutation matrices find $P^{-1}$ by trial and error (with 1's and 0's)

\[
P = \V{0,0,1;0,1,0;1,0,0}, \ \ \ P = \V{0,1,0;0,0,1;1,0,0}
\]



\item[7]

If A has row 1 + row 2 = row 3 show that A is not invertible
\begin{enumerate}
\item
Explain why Ax = (1,0,0) cannot have a solution
\item
Which right sides $(b_1,b_2,b_3)$ might allow a solution to Ax = b
\item
what happens to row 3 in elimination?
\end{enumerate}


\item[13]
If the product $M = ABC$ of three square matrices is invertible. then B is invertible ( so are A and C), find a formula for $B^{-1}$ that involves $M^{-1}$ and A and C.
\item[22]
Change I into $A^{-1}$ as you reduce A to I by row operations
\[
\V{A,I} = \V{1,3,1,0;2,7,0,1}
\]
and
\[
\V{A,I} = \V{1,4,1,0;3,9,0,1}
\]


\item[23]
Follow the 3 by 3 text example but with plus signs in A, eliminate above and below the pivots to reduce $[A \ I ]$ to $[I \ A^{-1}]$


\item[27]
Invert these matrices by the Gauss Jordan method starting with [A \ I]
\[
A = \V{2,1,1;1,2,1;1,1,2}, \ \ \ A = \V{1,1,1;1,2,2;1,2,3}
\]

\item[29]

Exchange rows and continue with Gauss Jordan to find $A^{-1}$
\[
[A \ I ] = \V{0,2,1,0;2,2,0,1}
\]

\end{enumerate}

\newpage









\textbf{2.6}: 1,3,4,5,7,10,12,15



\begin{enumerate}

\item


\end{enumerate}


\newpage










\textbf{2.7}: 1,2,6,12,15,16


\begin{enumerate}

\item[1]
Forward elimination changes $\V{1.1;1,2} x = b$ into  triangular $\V{1,1;0,1}x = c$
\[
x + y = 5 , \ \ x +2y = 7 \rightarrow x+y = 5, \ \ y = 2
\]
\[
\V{1,1,5;1,2,7} \rightarrow \V{1,1,5;0,1,2}
\]

That step subtracted $l_{21}$ = BLANK times row 1 from row 2.  The reverse step adds $l_{21}$ times row 1 to row 2 .  The matrix for that reverse step is $L = BLANK$.  Multiply this L times the triangular system $\V{1,1;0,1}x_1 = \V{5;2}$ to get BLANK = BLANK.  In letters L multiplies Ux = c to give BLANK


\item[2]
Write down the 2 by 2 triangular systems Lc= b and Ux =c from problem 1.  Check that c = (5,2) solves the first one.  Find the x that solves the second one.


\item[6]
What two elimination matrices $E_{21}$ and $E_{32}$ put A into upper triangular form $E_{32}E_{21} A = U$?  Multiply by 



\item[12]



\item[15]



\item[16]

\end{enumerate}

















\end{document}
